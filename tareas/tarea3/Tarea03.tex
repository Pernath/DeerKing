%Especificacion
\documentclass[12pt]{article}

%Paquetes
\usepackage[left=2cm,right=2cm,top=3cm,bottom=3cm,letterpaper]{geometry}
\usepackage{lmodern}
\usepackage[T1]{fontenc}
\usepackage[utf8]{inputenc}
%\usepackage[spanish,activeacute]{babel}
\usepackage{mathtools}
\usepackage{amssymb}
\usepackage{enumerate}
%\usepackage{tabularx}
%\usepackage{wasysym}
\usepackage{graphicx}
%\graphicspath { {tarea01/media/} }
%\usepackage{pifont}

%Preambulo
\title{Complejidad Computacional \\ Tarea 2.1}
\author{Karla Adriana Esquivel Guzmán \\ Andrea Itzel González Vargas\\ Luis Pablo Mayo Vega \\ Carlos Gerardo Acosta Hernández}
\date{Entrega: 03/04/17 \\ Facultad de Ciencias UNAM}

\setlength\parindent{0pt}

\begin{document}
\maketitle
\section*{Ejercicios}
\subsubsection*{1. Demuestra que el lenguaje $\Sigma_iSAT$ es completo para $\Sigma^P_i$ bajo reducciones polinomiales
temporales. Recuerda que $SAT$ es $NP-completo$.}
\subsubsection*{2. Demuestra que si $3SAT$ es temporalmente reductible polinomialmente a $\overline{3SAT}$ entonces $PH = NP$.}
Sabemos que $3SAT$ es $NP-completo$, entonces $\overline{3SAT} \in coNP$. \\
Supongamos que $3SAT$ es reductible a $\overline{3SAT}$, esto implica que $NP = coNP$. Como $\sum_1^p = NP$ y $\prod_1^p = coNP$, entonces $\sum_1^p = \prod_1^p$. Como vimos en clase, para toda $i \geq 1$ si $\sum_i^p = \prod_i^p$ entonces $PH = \sum_i^p$, o sea que la jerarquía se colapsa al nivel $i$. Como $\sum_1^p = \prod_1^p$ entonces $PH = \sum_1^p = NP$. \\
Por lo tanto si $3SAT$ es reductible a $\overline{3SAT}$ (o sea $NP = coNP$), entonces $PH = NP$.

\subsubsection*{3. Demuestra que si $P^A = NP^A$ (para algún lenguaje $A$), entonces $PH^A \subseteq P^A$ .}
\subsubsection*{4. Demuestra que si $EXP \subseteq P/poli$, entonces $EXP = \Sigma^p_2$ .}

Todos los ejercicios son sacados de la Internet, con variaciones.
\end{document}