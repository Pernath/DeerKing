%Especificacion
\documentclass[12pt]{article}

%Paquetes
\usepackage[left=2cm,right=2cm,top=3cm,bottom=3cm,letterpaper]{geometry}
\usepackage{lmodern}
\usepackage[T1]{fontenc}
\usepackage[utf8]{inputenc}
%\usepackage[spanish,activeacute]{babel}
\usepackage{mathtools}
\usepackage{amssymb}
\usepackage{enumerate}
%\usepackage{tabularx}
%\usepackage{wasysym}
\usepackage{graphicx}
%\graphicspath { {tarea01/media/} }
%\usepackage{pifont}

%Preambulo
\title{Complejidad Computacional \\ Tarea 2.1}
\author{Karla Adriana Esquivel Guzmán \\ Andrea Itzel González Vargas\\ Luis Pablo Mayo Vega \\ Carlos Gerardo Acosta Hernández}
\date{Entrega: 03/04/17 \\ Facultad de Ciencias UNAM}

\setlength\parindent{0pt}

\begin{document}
\maketitle
\section*{Ejercicios}
\subsubsection*{1. Demuestra que el lenguaje $\Sigma_iSAT$ es completo para $\Sigma^P_i$ bajo reducciones polinomiales
temporales. Recuerda que $SAT$ es $NP-completo$.}
\subsubsection*{2. Demuestra que si $3SAT$ es temporalmente reductible polinomialmente a $\overline{3SAT}$ entonces $PH = NP$.}
Sabemos que $3SAT$ es $NP-completo$, entonces $\overline{3SAT} \in coNP$. \\
Supongamos que $3SAT$ es reductible a $\overline{3SAT}$, esto implica que $NP = coNP$. Como $\sum_1^p = NP$ y $\prod_1^p = coNP$, entonces $\sum_1^p = \prod_1^p$. Como vimos en clase, para toda $i \geq 1$ si $\sum_i^p = \prod_i^p$ entonces $PH = \sum_i^p$, o sea que la jerarquía se colapsa al nivel $i$. Como $\sum_1^p = \prod_1^p$ entonces $PH = \sum_1^p = NP$. \\
Por lo tanto si $3SAT$ es reductible a $\overline{3SAT}$ (o sea $NP = coNP$), entonces $PH = NP$.

\subsubsection*{3. Demuestra que si $P^A = NP^A$ (para algún lenguaje $A$), entonces $PH^A \subseteq P^A$ .}
\subsubsection*{4. Demuestra que si $EXP \subseteq P/poli$, entonces $EXP = \Sigma^p_2$ .}
\underline{\textbf{\textit{Dem.}}}\\
Sea $L \in EXP$, entonces existe una máquina de Turing \textit{time-oblivious} $M$ que decide $L$ en tiempo $2^{p(n)}$ $p.a.$ polinomio $p$. Sea $s \in \{0,1\}^n$ una cadena de entrada para $M$. Sabemos por la definición de $M$ que para cada $i \in [2^{p(n)}]$ denotamos con $z_i$ la codificación de
la i-ésima ``instantánea'' de la ejecución de $M$ con la entrada $s$.
Como $EXP \subseteq P/poli$, entonces existe un circuito $C$ de tamaño $q(n)$ (p.a. polinomio $q$), tal que calcula $z_i$ a partir de una $i$.
La correctud de lo que calcula este circuito mencionado puede ser expresado como un predicado $coNP$. Así, \\
\begin{equation}
  s \in L \iff \exists C \in \{0,1\}^{q(n)}\;\; \forall i,i1,...,ik \in \{0,1\}^{p(n)}\;\; T(s,C(i),C(i_1),...,C(i_k)) = 1
\end{equation}
donde $T$ es una $TM$ que verifica esas condiciones en tiempo polinomial. Se puede entonces concluir que $L \in \Sigma^P_2$, que es lo que queremos.
Para probar esto, consideremos $p(n) = 2^{n^k}$. Consideremos cada entrada $(i,t)$ en la tabla de $M$, codifica una cadena $z_{i,t}$, \textit{i.e.}, el contenido
de la celda $i$, al momento $t$, siempre que la cabeza lectora esté en la entrada $i$ al momento $t$, y de ser así, $z$ almacena el estado interno
de $M$. Ahora consideremos
\begin{equation}
  L_M = \{\langle s, i, t, z\rangle \;\;\;|\;\;\; con\;\;la\;\;entrada\;\;s\;\;tenemos\;\; z_{i,t} = z \;\;\;para\;\; M \}
\end{equation}
Simulando $M$ tendremos que $L_M \in EXP \subseteq P/poli$. Utilizando
circuitos de tamaño polinomial para $L_M$, podemos construir un circuito
 de tamaño polinomial $C$  de múltiple salida, tal que $C(\langle s,i,t\rangle)$ = $z$. Como buscábamos en $(1)$, decimos entonces que:\\
\begin{quote}
%  \center
$  s \in L \iff \exists C\;\; \forall i,t\;\; t.q.\;\; C(\langle s,i,t\rangle)\;\; acepta\;\;
si$\\
$C(\langle s,i-t,t-1\rangle),\;\;C(\langle s,i,t-1\rangle),\;\;C(\langle s,i+1,t-t1\rangle)\;\; y\;\; C(\langle s,1,2^{n^k}\rangle)\;\; aceptan.$
\end{quote}
Por lo tanto si $EXP \subseteq P/poli$, entonces $EXP = \Sigma^p_2$.
\end{document}