''%Especificacion
\documentclass[12pt]{article}

%Paquetes
\usepackage[left=2cm,right=2cm,top=3cm,bottom=3cm,letterpaper]{geometry}
\usepackage{lmodern}
\usepackage[T1]{fontenc}
\usepackage[utf8]{inputenc}
\usepackage[spanish,activeacute]{babel}
\usepackage{mathtools}
\usepackage{amssymb}
\usepackage{enumerate}
\usepackage{xfrac}
%\usepackage{tabularx}
%\usepackage{wasysym}
\usepackage{graphicx}
%\usepackage{pifont}

%Preambulo
\title{Complejidad Computacional \\ Tarea 2.2}
\author{Karla Adriana Esquivel Guzmán \\ Andrea Itzel González Vargas\\ Luis Pablo Mayo Vega \\ Carlos Gerardo Acosta Hernández}
\date{Entrega: 02/05/17 \\ Facultad de Ciencias UNAM}

\setlength\parindent{0pt}

\begin{document}
\maketitle
\section*{Ejercicios}
\subsubsection*{1. Demuestra que si $P = NP$, entonces existe un lenguaje $EXP$ que requiere circuitos de tamaño $\sfrac{2^{n}}{n}$.} 
\subsubsection*{2. Describe un circuito $NC$ para calcular el producto de dos matrices de $n \times n$.}
\subsubsection*{3. Demuestra que $L$ es $P-completo$ implica que $L \in NC$ si y sólo si, $P=NC$ }
Primero, por definición, tenemos que $L$ es $P-completo \; sii \;$ $L \in P$ y $\forall L' \in P$ \[ L' \leq_{logspace} L \]
Además. sabemos que $L \in NC^i$ si $\exists c > 0$ tal que $L$ puede ser decidido por una familia de circuitos $\{C_n\}$ \textit{logspace uniformes} con tamaño $\mathcal{O}(n^c)$ y profundidad $\mathcal{O}(log^in)$.\\
La clase de Nick es $NC = \displaystyle\bigcup_{i \leq 1} NC^i$ \\
Como $\{C_n\}$ es \textit{logspace uniforme} cumple las siguientes propiedades.
\begin{enumerate}
\item $C_i$ contiene solo circuitos $\land, \not, \lor$
\item tiene $i$ entradas
\item tiene tamaño $\mathcal{O}(n^c)$
\item tiene profundidad $log^ci$
\item hay una $\mathsf{M \in TM}$ que calcula en tiempo polinomial el circuito $c_i$ a partir de la entrada $1^i$
\end{enumerate}
\end{document}
