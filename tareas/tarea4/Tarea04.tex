%Especificacion
\documentclass[12pt]{article}

%Paquetes
\usepackage[left=2cm,right=2cm,top=3cm,bottom=3cm,letterpaper]{geometry}
\usepackage{lmodern}
\usepackage[T1]{fontenc}
\usepackage[utf8]{inputenc}
%\usepackage[spanish,activeacute]{babel}
\usepackage{mathtools}
\usepackage{amssymb}
\usepackage{enumerate}
%\usepackage{tabularx}
%\usepackage{wasysym}
\usepackage{graphicx}
%\usepackage{pifont}
\usepackage{qtree}
\newcommand*{\QEDB}{\hfill\ensuremath{\square}}%

%Preambulo
\title{Complejidad Computacional \\ Tarea 2.2}
\author{Karla Adriana Esquivel Guzmán \\ Andrea Itzel González Vargas\\ Luis Pablo Mayo Vega \\ Carlos Gerardo Acosta Hernández}
\date{Entrega: 25/04/17 \\ Facultad de Ciencias UNAM}

\setlength\parindent{0pt}

\begin{document}
\maketitle
\section*{Ejercicios}
\subsubsection*{1. Demuestra que si $P = NP$, entonces existe un lenguaje $EXP$ que requiere circuitos de tamaño ${2^{n}}/{n}$.} 
\textbf{Demostración:}
Por el teorema de Meyer (teorema 6.20 del libro) Implica que si $P=NP$ entonces $Exp$ $\not \subset$ $P/Poly$.\\
Entonces como $Exp$ $\not\in$ $P/Poly$ por lo tanto debe existir un lenguaje $L$ $\in$ $EXP$ que requiere un circuito de tamaño $2^n$/$n$.
\subsubsection*{2. Describe un circuito $NC$ para calcular el producto de dos matrices de $n \times n$.}
\textbf{Demostración:}
Primero veamos el caso para matrices $2x2$ \\

\[A= \left(
\begin{array}{cc}
\textsubscript{a} & \textsubscript{b}  \\
\textsubscript{c} & \textsubscript{d}  \\
\end{array} 
\right)\]

\[B= \left(
\begin{array}{cc}
\textsubscript{f} & \textsubscript{g}  \\
\textsubscript{h} & \textsubscript{i}  \\
\end{array} 
\right)\]

\[AB= \left(
\begin{array}{cc}
\textsubscript{af + bh} & \textsubscript{ag + bj}  \\
\textsubscript{cf + dh} & \textsubscript{cg + dj}  \\
\end{array} 
\right)\] 

\newpage

Para cada entrada de la matriz AB la calcularemos con el siguiente circuito \\



\Tree [.$\vee$ [.{\sc $\wedge$} {\bf a} f ] [.{\sc $\wedge$} {\bf b} h ]  ]\\



Ahora veamos para una matriz de $3x3$ \\

\[A= \left(
\begin{array}{ccc}
\textsubscript{a} & \textsubscript{b} & \textsubscript{c} \\
\textsubscript{d} & \textsubscript{e} & \textsubscript{f} \\
\textsubscript{g} & \textsubscript{h} & \textsubscript{i} \\
\end{array} 
\right)\]

\[B= \left(
\begin{array}{ccc}
\textsubscript{l} & \textsubscript{m} & \textsubscript{n} \\
\textsubscript{o} & \textsubscript{p} & \textsubscript{q} \\
\textsubscript{r} & \textsubscript{s} & \textsubscript{t} \\
\end{array} 
\right)\]

\[AB= \left(
\begin{array}{ccc}
\textsubscript{al + bo + cr} & \textsubscript{...} & \textsubscript{...} \\
\vdots & \vdots & \vdots \\
\end{array} 
\right)\] \\

El circuito quedaría\\

Para la entrada [1,1] el valor de las hojas.\\


    \Tree[.$\vee$ [.$\vee$ [.$\wedge$ a f ] 
      [.$\wedge$ b o ] ] [.$\vee$ [.$\wedge$ c r ] ] ] \\
      
      
Podemos generalizar un árbol para las matrices de nxn que tengan profundidad de $log2n$ donde las primeras $(log2n)-1$ niveles son para ir dividiendo por la mitad los OR (Sumas) que se deben hacer para obtener el valor de la entrada y el último nivel hace el AND (Multiplicación).\\


\newpage

\subsubsection*{3. Demuestra que $L$ es $P-completo$ implica que $L \in NC$ si y sólo si, $P=NC$ }

\fbox{\textbf{$\Rightarrow$}} Suponemos que \textbf{$L \in NC$} \\

\textbf{P.D.} $P = NC$ \\

$L \in NC$ y es $P-completo$, por lo que
  \begin{gather}
    \forall L' \in P,\  L' \leq_{NC} L
  \end{gather}
  i.e. toda $L'$ en $P$ es $NC$ reductible a $L$. Por otro lado tenemos que
  \begin{gather}
    si\ \ L_1 \leq_{NC} L_2\ \ y\ \ L_2 \in NC,\ \ entonces\ \ L_1 \in NC    
  \end{gather}
  Por (1) y (2) sabemos entonces que $\forall L' \in P$, $L' \in NC$. Como todo $L'$ en $P$ pertenece también a $NC$ (i.e. $P \subseteq NC$) y $NC \subseteq P$, entonces $P = NC$. \\

  \fbox{\textbf{$\Leftarrow$}} Suponemos que \textbf{$P = NC$} \\

  \textbf{P.D.} L $\in$ NC \\

  L es $P-completo$, entonces $L \in P$. Ya que $P = NC$, entonces $L \in NC$.
  \begin{flushright}
    \QEDB
  \end{flushright}

\end{document}
