%Especificacion
\documentclass[12pt]{article}

%Paquetes
\usepackage[left=2cm,right=2cm,top=3cm,bottom=3cm,letterpaper]{geometry}
\usepackage{lmodern}
\usepackage[T1]{fontenc}
\usepackage[utf8]{inputenc}
%\usepackage[spanish,activeacute]{babel}
\usepackage{mathtools}
\usepackage{amssymb}
\usepackage{enumerate}
\usepackage{enumitem}
%\usepackage{tabularx}
%\usepackage{wasysym}
\usepackage{graphicx}
%\usepackage{pifont}
\usepackage{qtree}
\newcommand*{\QEDB}{\hfill\ensuremath{\square}}%

%Preambulo
\title{Complejidad Computacional \\ Tarea 3}
\author{Karla Adriana Esquivel Guzmán \\ Andrea Itzel González Vargas\\ Luis Pablo Mayo Vega \\ Carlos Gerardo Acosta Hernández}
\date{Entrega: 02/05/17 \\ Facultad de Ciencias UNAM}

\setlength\parindent{0pt}

\begin{document}
\maketitle
\section*{Ejercicios}
\subsubsection*{1.}
Sea $L \in \{0,1\}^*$ tal que existe una PTM con tiempo polinomial de ejecución tal que $\forall \; \alpha \in \{0,1\}^*$

\begin{enumerate}[label=\alph*)]
\item Si $\alpha \in L$ entonces $Pr[M(\alpha) = 1] \geq n^{-c}$
\item Si $\alpha \notin L$ entonces $Pr[M(\alpha) = 1] = 0$
\end{enumerate}
Demuestra que para todo $d > 0$ existe una $M' \in PTM$ con tiempo polinomial de ejecución tal que $\forall \;\alpha \in \{0,1\}^*$:
\begin{enumerate}[label=\alph*)]
\item Si $\alpha \in L$ entonces $Pr[M(\alpha) = 1] \geq 1 - 2^{-n^d}$;
\item Si $\alpha \notin L$ entonces $Pr[M(\alpha) = 1] = 0$
\end{enumerate}
\textsf{\textbf{Demostración:}}\\
\textbf{Teorema (Cota de Chernorff)}: Sean $X_1, \dots, X_n$ un conjunto de variables booleanas aleatorias tales que cada $X_i$ es un bit igual a 1 con probabilidad $\leq p$. Entonces:
\[Pr[|\Sigma_{i=1}^k X_i - pk | > \delta pk] < e^{-{\dfrac{\delta^2}4}pk}.\]
para una $\delta$ lo suficientemente pequeña. \\
\textit{M'} simulará al \textit{M} con la entrada $\alpha$ $k=n^{2c+d}$ veces. \textit{M'} acepta si al menos una de las simulaciones acepta, rechaza si todas las simulaciones rechazan.\\
Sean $X_1, \dots, X_k$ variables booleanas independientes con $E[X_i] = Pr[X_i=1] \geq p \; con \; p=n^{-c}$. Usando la configuración $\delta = n ^{-c}/2$ garantizamos que si $\Sigma_{i=1}^k X_i \geq pk - \delta pk$ entonces obtendremos una respuesta correcta. Así acotamos la probabilidad de obtener una respuesta incorrecta a
\[ e^{-{\dfrac{1}{4|x|^{2c}}}\dfrac{1}{2}|x|^{2c+d}} \leq 2^{-n^d} \]

\subsubsection*{2.}
\subsubsection*{3.}
\end{document}
