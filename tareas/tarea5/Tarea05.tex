%Especificacion
\documentclass[12pt]{article}

%Paquetes
\usepackage[left=2cm,right=2cm,top=3cm,bottom=3cm,letterpaper]{geometry}
\usepackage{lmodern}
\usepackage[T1]{fontenc}
\usepackage[utf8]{inputenc}
%\usepackage[spanish,activeacute]{babel}
\usepackage{mathtools}
\usepackage{amssymb}
\usepackage{enumerate}
\usepackage{enumitem}
%\usepackage{tabularx}
%\usepackage{wasysym}
\usepackage{graphicx}
%\usepackage{pifont}
\usepackage{qtree}
\newcommand*{\QEDB}{\hfill\ensuremath{\square}}%

%Preambulo
\title{Complejidad Computacional \\ Tarea 3}
\author{Karla Adriana Esquivel Guzmán \\ Andrea Itzel González Vargas\\ Luis Pablo Mayo Vega \\ Carlos Gerardo Acosta Hernández}
\date{Entrega: 02/05/17 \\ Facultad de Ciencias UNAM}

\setlength\parindent{0pt}

\begin{document}
\maketitle
\section*{Ejercicios}
\subsubsection*{1. Sea $L \subseteq \{0,1\}^*$ tal que existe una \textit{PTM} con tiempo polinomial de ejecución tal que $\forall \alpha \in \{0,1\}^*$:\vspace{0.3cm}
  \begin{enumerate}[label=\alph*)]
  \item Si $\alpha \in L$ entonces $Pr[M(\alpha) = 1] \geq n^{-1}$
    \item Si $\alpha \notin L$ entonces $Pr[M(\alpha) = 1] = 0$
  \end{enumerate}
  Demuestra que para todo $d > 0$, existe una $M' \in PTM$ con tiempo polinomial de ejecución tal que $\forall \alpha \in {0,1}^*$:
  \begin{enumerate}[label=\alph*)]
  \item Si $\alpha \in L$ entonces $Pr[M'(\alpha) = 1] \geq 1-2^{-n^d}$
  \item Si $\alpha \notin L$ entonces $Pr[M'(\alpha) = 1] = 0$
  \end{enumerate}
}

\subsubsection*{2. Demuestra que \textit{ZPP} = \textit{RP} $\cap$ \textit{coRP}}
\subsubsection*{3. Demuestra que \textit{BPL} $\subseteq$ \textit{P}, donde \textit{BPL} es la clase de lenguajes decidibles por \textit{PTM} que utilizan espacio logarítmico.}
\textbf{Dem.}\\
Sea $L$ un lenguaje \textbf{\textit{BPL}} y $M$ una $TM$ tal que para una entrada $x \in L$, $Pr[M(x) = L(x)] \leq \frac{2}{3}$. Supongamos que la cadena $x$ es de longitud $n$, ahora bien, sea $C$ el número de configuraciónes de $M(\cdot,x)$, entonces $S = C \times C$ es una matriz tal que $S_{{c_1},{c_2}} = \frac{1}{2}$ si la configuración $c_2$ es obtenida en un sólo paso de $M$ desde $c_1$ y  $S_{{c_1},{c_2}} = 0$ en cualquier otro caso. $\forall t$, la entrada $S^t_{{c_1},{c_2}}$ nos dice la probabilidad de obtener
la configuración $c_2$ a partir de $c_1$ en $t$ pasos, donde $S^t$ es la matriz resultante de la multiplicación de $S$ con ella misma, $t$ veces. Al calcular todas las potencias de $S$ hasta el tiempo de ejecución de $M(\cdot,x)$, podemos calcular la probabilidad de de aceptación para $M(r,x)$ y decidir si $x \in L$. Notemos que cada probabilidad es un entero múltiplo de $\frac{1}{2^{p(n)}}$, de manera que puede ser representada con un número polinomial de dígitos. Por lo tanto \textit{BPL} $\subseteq$ \textit{P}.


\end{document}
