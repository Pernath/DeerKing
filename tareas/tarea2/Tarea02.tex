%Especificacion
\documentclass[12pt]{article}

%Paquetes
\usepackage[left=2cm,right=2cm,top=3cm,bottom=3cm,letterpaper]{geometry}
\usepackage{lmodern}
\usepackage[T1]{fontenc}
\usepackage[utf8]{inputenc}
%\usepackage[spanish,activeacute]{babel}
\usepackage{mathtools}
\usepackage{amssymb}
\usepackage{enumerate}
%\usepackage{tabularx}
%\usepackage{wasysym}
\usepackage{graphicx}
%\graphicspath { {tarea01/media/} }
%\usepackage{pifont}

%Preambulo
\title{Complejidad Computacional \\ Tarea 2}
\author{Karla Adriana Esquivel Ramírez \\ Andrea Itzel González Vargas\\ Luis Pablo Mayo Vega \\ Carlos Gerardo Acosta Hernández}
\date{Entrega: 14/03/17 \\ Facultad de Ciencias UNAM}

\begin{document}
\maketitle
\section*{Ejercicios}

\subsubsection*{1.}

\subsubsection*{2.}

\subsubsection*{3.}
\textbf{Demuestra que el lenguaje} \\
\indent \textit{SPACETM} = $\{ \langle$M$\rangle \langle \alpha 
\rangle 1^n$ | M es una MT que acepta $\alpha$ en espacio \textit{n} $\}$ \\
\textbf{es PSPACE-completo.} \\

 P.D.\\


I) $SPACETM \in PSPACE.$\\

II) Cualquier $L \in PSPACE$ se puede reducir a SPACETM\\
 
Entrada < M, \alpha, $1^{n}$ >\\

Construimos $M’\in TM$ para Simular M, M’ con dos contadores. \\

I) Un contador para contar el espacio utilizado por M, el otro contador contará el número de pasos que va a ejecutar M. Cada vez después de que M ' Simule un paso de M, actualiza los contadores y comprueba si el espacio utilizado 
es mayor que n o si las etapas que se ejecutan en M son mayores que $2^{cn}$  para alguna constante c que está relacionada con M. 

Si cualquiera de las anteriores ocurre entonces se rechaza inmediatamente.  Observemos que el número total de configuraciones posibles de M es como máximo $2^{cn}$, por lo tanto, si M no se detiene en este número de pasos, entonces debe haber entrado en un bucle infinito y M ' Rechaza, después de simular M corriendo en w, si M acepta w en espacio n, M ' Acepta, en otro caso M’ rechaza.
Por definición de SPACETM, M’ decide a SPACE. Para simular M solamente utilzamos espacio O(n). Entonces SPACETM ∈ PSPACE\\

II) Ahora Observemos que para cualquier $L \in PSPACE$ dónde L es un lenguaje existe una Maquina de Turing Determinista M que decide L usando espacio s(n) el cual es una entrada polinomial de tamaño n. Consideremos la función f: $\big \{0,1}\big \}^{n}$ \Rightarrow $\big \{0,1}\big \}^{m}$ tal que
$f(x)=< M, x, 1^{s(n)}>.$ Dónde m es polinomial de n. La función f puede ser computada en espacio y tiempo polinomial. Por definición de SPACETM $x\in L$ syss $f(x)\in SPACETM.$ Así que SPACETM es PSPACE-hard entonces $SPACETM \in PSPACE-completo$ \\


\therefore $ Es PSPACE-Completo.$\\



$\dots$

\subsubsection*{4.}
\textbf{Demuestra $2SAT \in NL$ } \\


Primero veamos que para todo F en 2-SAT podemos construir una grafica dirigida G de la siguiente manera:

	V(G) son las literales de F y hay una arista de a a b en E(G) si y solo si (a’ v b ) esta e F 

Con esta construccion veamos que:

	 F es satisfaceible si y solo si G no tiene un ciclo donde exista una literal y su negacion.

=>)

veamos que a’ V b es equivalente a a->b

Supongamos que tenemos un ciclo que contiene a b y b’ en G. 
Si esto pasa entonces por transitividad signfica que b→...→b’ y que b’→…→b por lo que tendriamos  b ↔ b’ lo cual es una contradiccion.

Por lo que si F es satisfacible entonces G no tiene un ciclo donde exista una literal y su negacion.

<=)

Veamos estos haciend induccion sobre el numero de variables 
Cuando se tienen 0 varaibles

Si F no tiene variables entonces F es satisfacible y G no tiene ciclos.

Hip. Ind

Supongamos que se cumple cuando se tiene n variables

Ahora vemos que pasa cuadno se tiene un variables mas:

Tomamos una literal b que en la grafica G generada no tiene un camino de b a b’.

Ahora supongamos que hay un camino de b a c y un camino de b a c’

Por contraposicion tendriamos un camino de c’ a b’ y por transitividad tendiamos un camino  b a b’ lo cual sera una contradiccion.

Ahora hagamos todas las literales  a las que llegamos a partir de b verdaderas, esta conjunto esta bien definido por lo anterior visto. Entonces todas las clausulas que contenga a un literal de este conjunto se satisfacen. Y nos queda una grafica G’ dada por las clausulas restantes las cuales tampoco tienen un ciclo con una literal y su negacion y por Hipotesis de induccion esta formula tambien es satisfacible.

.: Si G no tiene ciclos donde existe una literal y su negacion F es satisfacible




En libro tenemos que NL= coNL veamos que 2-SAT’ en NL implica 2-SAT en NL

Para ver si algo cumple con 2-SAT’ basta con ver si hay una camino de alguna literal b a b’ y de b’ a b esto es que exista un ciclo en la grafica G generada. Por lo que podemos reducir el problema a PATH que de igual manera por lo visto en el libro PATH esta en NL.

.: 2-SAT’ esta en NL 
.: 2-SAT esta en NL

\subsubsection*{5.}

\subsubsection*{6.}
\end{document}