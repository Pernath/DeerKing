%Especificacion
\documentclass[12pt]{article}

%Paquetes
\usepackage[left=2cm,right=2cm,top=3cm,bottom=3cm,letterpaper]{geometry}
\usepackage{lmodern}
\usepackage[T1]{fontenc}
\usepackage[utf8]{inputenc}
%\usepackage[spanish,activeacute]{babel}
\usepackage{mathtools}
\usepackage{amssymb}
\usepackage{enumerate}
%\usepackage{tabularx}
%\usepackage{wasysym}
\usepackage{graphicx}
%\graphicspath { {tarea01/media/} }
%\usepackage{pifont}

%Preambulo
\title{Complejidad Computacional \\ Tarea 2}
\author{Karla Adriana Esquivel Ramírez \\ Andrea Itzel González Vargas\\ Luis Pablo Mayo Vega \\ Carlos Gerardo Acosta Hernández}
\date{Entrega: 14/03/17 \\ Facultad de Ciencias UNAM}

\begin{document}
\maketitle
\section*{Ejercicios}

\subsubsection*{1.}

\subsubsection*{2.}

\subsubsection*{3.}
\textbf{Demuestra que el lenguaje} \\
\indent \textit{SPACETM} = $\{ \langle$M$\rangle \langle \alpha 
\rangle 1^n$ | M es una MT que acepta $\alpha$ en espacio \textit{n} $\}$ \\
\textbf{es PSPACE-completo.} \\

\noindent \textbf{Demostración: \\}
Queremos ver que todo problema de decisión en espacio \textit{P} $\in$ \textit{PSPACE} puede ser reducido al Teorema de la Jerarquía Espacial en tiempo polinomial.

$\dots$

\subsubsection*{4.}

\subsubsection*{5.}

\subsubsection*{6.}
\end{document}