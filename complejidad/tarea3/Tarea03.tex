%Especificacion
\documentclass[12pt]{article}

%Paquetes
\usepackage[left=2cm,right=2cm,top=3cm,bottom=3cm,letterpaper]{geometry}
\usepackage{lmodern}
\usepackage[T1]{fontenc}
\usepackage[utf8]{inputenc}
%\usepackage[spanish,activeacute]{babel}
\usepackage{mathtools}
\usepackage{amssymb}
\usepackage{enumerate}
%\usepackage{tabularx}
%\usepackage{wasysym}
\usepackage{graphicx}
%\usepackage{pifont}

%Preambulo
\title{Complejidad Computacional \\ Tarea 2.1}
\author{Karla Adriana Esquivel Guzmán \\ Andrea Itzel González Vargas\\ Luis Pablo Mayo Vega \\ Carlos Gerardo Acosta Hernández}
\date{Entrega: 25/04/17 \\ Facultad de Ciencias UNAM}

\setlength\parindent{0pt}

\begin{document}
\maketitle
\section*{Ejercicios}
\subsubsection*{1. Demuestra que el lenguaje $\Sigma _iSAT$ es completo para $\Sigma ^P_i$ bajo reducciones polinomiales
temporales. Recuerda que $SAT$ es $NP-completo$.}
\textbf{Demostración:}
Primero, definimos  
\begin{equation*} \label{eq:1}
 \Sigma _{i}SAT = \{ \langle \varphi \rangle : \exists u_1 \forall u_2 \exists \dots Q_{i}u_{i}\varphi(u_1,u_2,\dots,u_{i}) = 1 \}
\end{equation*}
donde $Q_{i}$ es un cuantificador($\exists$ o $\forall$ dependiendo de la paridad de $i$), $\varphi$ es una fórmula booleana, y cada $u_k$ es un vector de variables booleanas.\\

Para $\Sigma_1SAT$, observemos que
\begin{enumerate}
\item $\Sigma_1SAT$ = $\{ \langle \varphi \rangle : \exists u_1 \varphi(u_1) = 1 \}$ 
\item $\Sigma^P_1 = NP$
\end{enumerate}
Así que $\Sigma _1SAT = SAT$ y ya sabemos que SAT es $NP-completo$. Entonces, en este caso, $\Sigma_1SAT$ es $\Sigma^P_1-completo$. \\

Ahora queremos demostrar que $\Sigma _{i}SAT$ es $\Sigma^P_{i}-completo$, para $i > 1$.\\
Recordemos que podemos definir a las clases $\Sigma_i$ con Máquinas de Turing con Oráculo de la siguiente manera: \\
\begin{equation*}
\Sigma_i = NP^{\Sigma_{i-1}}
\end{equation*} 
Observemos que en el caso i = 2 tenemos \\
\begin{equation*}
\Sigma_2 = NP^{NP} = NP^{SAT} = NP^{\Sigma_1SAT}
\end{equation*}
Probemos que $\Sigma_i = NP^{\Sigma_{i-1}SAT}$ usando cuantificadores, a partir del caso $i=2$.\\

\fbox{\textbf{$\subseteq$}} Tomemos $L \in \Sigma_2$ así que $\exists p$ un polinomio y $M \in TM \; polinomial$ tal que
\begin{equation*}
x \in L \iff \exists v_1 \in \{ 0,1 \}^{p(|x|)} \forall v_2 \in \{ 0,1 \}^{p(|x|)} M(x,u_1,u_2) = 1  
\end{equation*}
Sea $\bar{L}$ el complemento de $L$, se tiene que $\bar{L} \in \Pi_2$ así que
\begin{equation*}
y \in \bar{L} \iff \forall w_1 \in \{ 0,1 \}^{q(|y|)} \exists w_2 \in \{ 0,1 \}^{q(|y|)} \bar{M}(y,w_1,w_2) = 1
\end{equation*}
Definimos $L'$ de la siguiente manera
\begin{equation*}
\langle y,w_1 \rangle \in L' \iff \exists w_2 \in \{ 0,1 \}^{q(|y|)} M(y,w_1, w_2) = 1
\end{equation*}
Así que $L' \in \Sigma_1$. Luego $\exists f$ una reducción polinomial tal que $\langle y, w_1 \rangle \in L' \iff f(y,w_1) \in \Sigma_1SAT$, es decir, $f(y,w_1)$ ($\varphi(z)$) es una fórmula en $\Sigma_1SAT$, así $\exists w_2$ tal que $\varphi(w_2) = 1$. \\
Construyamos $m \in TM$ de la siguiente forma: \\
$m = \; " \; Con \; entrada \; \langle x \rangle$ 
\begin{enumerate}
\item Construyamos (no deterministamente) a $w_1$.
\item Usemos $f$ para construir $\varphi(x,w_1)(z)$.
\item Consultemos a $\Sigma_1SAT$ si $\varphi(x,w_1)(z)$.
\item Si $\Sigma_1SAT$ rechaza, acepto.
\item Rechazo en cualquier otro caso.
\end{enumerate}
\fbox{\textbf{$\supseteq$}} Ahora probemos que $NP^{SAT} \in \Sigma_2$.\\
Sea $L \in NP^{SAT}$, intuitivamente, si \textit{L} puede ser decidido en tiempo polinomial por una Máquina de Turing con Oráculo \textit{M} entonces podríamos decir que existen elecciones no deterministas \textit{y}, consultas al oráculo $q_1, \dots, q_k$ y respuestas del oráculo $ a_1, \dots, a_k$ tales que \textit{M} acepta una entrada \textit{x} en tiempo polinomial, lo que implicaría que $NP = coNP$. La falla de este razonamiento está en que debemos incluir una condición que requiera que las respuestas del oráculo ($ a_1, \dots, a_k$) sean válidas para las consultas $q_1, \dots, q_k$ \textit{sii} $q_j \in SAT$. Así, podemos describir a \textit{L} si observamos que $x \in L \iff \exists y, q_1,\dots,q_k,a_1,\dots,a_k$ tales que \textit{M} acepta \textit{x} y $a_j = 1 \iff q_j \in SAT$. Sin embargo, decidir la relación "\textit{M} acepta \textit{x} $a_j = 1 \iff q_j \in SAT$ " requiere decidir $SAT$, así que nos gustaría reescribir esta relación en términos de una relación de tiempo \textit{polinomial}. Podemos hacer esto si observamos que $q_j \in SAT \iff \exists x^Y_j$ tal que $q_j(x^Y_j) = 1$ y $q_j \notin SAT \iff \forall x^N_j, q_j(x^N_j) = 0$. De este modo, reescribimos la caracterización de \textit{L} del modo siguiente
\begin{equation*}
x \in L \iff \exists y, q_1,\dots,q_k,a_1,\dots,a_k, x^Y_1,\dots,x^Y_k \; tales  \; que \; \forall x^N_1,\dots,x^N_k 
\end{equation*}
\begin{itemize}
\item \textit{M} acepta \textit{x}.
\item $a_j = 1 \implies q_j(x^Y_j) = 1$.
\item $a_j = 0 \implies q_j(x^N_j) = 0$.
\end{itemize}

\subsubsection*{2. Demuestra que si $3SAT$ es temporalmente reductible polinomialmente a $\overline{3SAT}$ entonces $PH = NP$.}
Sabemos que $3SAT$ es $NP-completo$, entonces $\overline{3SAT} \in coNP$. \\
Supongamos que $3SAT$ es reductible a $\overline{3SAT}$, esto implica que $NP = coNP$. Como $\sum_1^p = NP$ y $\prod_1^p = coNP$, entonces $\sum_1^p = \prod_1^p$. Como vimos en clase, para toda $i \geq 1$ si $\sum_i^p = \prod_i^p$ entonces $PH = \sum_i^p$, o sea que la jerarquía se colapsa al nivel $i$. Como $\sum_1^p = \prod_1^p$ entonces $PH = \sum_1^p = NP$. \\
Por lo tanto si $3SAT$ es reductible a $\overline{3SAT}$ (o sea $NP = coNP$), entonces $PH = NP$.

\subsubsection*{3. Demuestra que si $P^A = NP^A$ (para algún lenguaje $A$), entonces $PH^A \subseteq P^A$ .}
Tenemos que si $P^A$ = $NP^A$ entonces $P^A$ es cerrado bajo el complemento => $P^A$ = $coNP$, ahora de manera concisa tenemos que
$P^A$ = $\Sigma_{1}$ $P^A$ = $\sqcap_{1}$ $P^A$. \\

Ahora Demostremos por Inducción que si $P^A$ = $\Sigma_{1}$ $P^A$ = $\sqcap_{1}$ $P^A$ => $P^A$ = $\Sigma_{i+1}$ $P^A$ = $\sqcap_{i+1}$ $P^A$\\
- Consideremos una $\Sigma_{i+1}$ $P^A$   M$\in$TM, que consiste en una serie de ramificaciones seguidas por una serie de ramificaciones universales.\\
- Consideremos ahora los subarboles de la trayectoria de un calculo cuyas raices son el primer paso universal a lo largo del camino para cada uno
de estos subarboles, M esta realizando un calculo $\sqcap_{i}$.\\
Por Hipotesis $\sqcap_{1}$ $P^A$ = $P^A$ así podemos reemplazar cada uno de estos subarboles de calculo por un método determinista en calculo de tiempo polinomial para formar una nueva Maquina S.\\
- Si dejamos que a(n) sea el máximo número de pasos dados por la Maquina alterna antes de que comiencen las ramas universales y P(n) sea el numero de pasos dados por cualquiera de las maquinas $P^A$ deterministas, que hemos sustituido por los calculos para $\sqcap_{i}$, entonces el tiempo en que corre S esta limitado por a(n) + $P^A$(a(n)).\\
Observemos que $P^A$(a(n)) es una composición de funciones, porque los subprocedimientos en $P^A$ estan calculando entradas que pueden ser mas grandes que n (Pero deberían ser mas pequeñas o igual a a(n) ya que solo se han ejecutado a(n) pasos en el tiempo que se usan los subprocedimientos.\\
-Como a y p son polinomiales también lo es su composicion, Por lo tanto S esta en $NP^A$, por Hipotesis $P^A$ = $NP^A$ => S esta en $P^A$\\
-De forma similar puede ser usado para reducir una maquina $\sqcap_{i+1}$ $P^A$ a una coNP M$\in$MT y Como $P^A$=$\Sigma_{i}$ $P^A$, poniendolo asi en $P^A$ también completamos el colapso de la jerarquía.\\
Por lo tanto $PH^A$ $\subseteq$ $PH^A$ 


\subsubsection*{4. Demuestra que si $EXP \subseteq P/poli$, entonces $EXP = \Sigma^p_2$ .}
\underline{\textbf{\textit{Dem.}}}\\
Sea $L \in EXP$, entonces existe una máquina de Turing \textit{time-oblivious} $M$ que decide $L$ en tiempo $2^{p(n)}$ $p.a.$ polinomio $p$. Sea $s \in \{0,1\}^n$ una cadena de entrada para $M$. Sabemos por la definición de $M$ que para cada $i \in [2^{p(n)}]$ denotamos con $z_i$ la codificación de
la i-ésima ``instantánea'' de la ejecución de $M$ con la entrada $s$.
Como $EXP \subseteq P/poli$, entonces existe un circuito $C$ de tamaño $q(n)$ (p.a. polinomio $q$), tal que calcula $z_i$ a partir de una $i$.
La correctud de lo que calcula este circuito mencionado puede ser expresado como un predicado $coNP$. Así, \\
\begin{equation}
  s \in L \iff \exists C \in \{0,1\}^{q(n)}\;\; \forall i,i1,...,ik \in \{0,1\}^{p(n)}\;\; T(s,C(i),C(i_1),...,C(i_k)) = 1
\end{equation}
donde $T$ es una $TM$ que verifica esas condiciones en tiempo polinomial. Se puede entonces concluir que $L \in \Sigma^P_2$, que es lo que queremos.
Para probar esto, consideremos $p(n) = 2^{n^k}$. Consideremos cada entrada $(i,t)$ en la tabla de $M$, codifica una cadena $z_{i,t}$, \textit{i.e.}, el contenido
de la celda $i$, al momento $t$, siempre que la cabeza lectora esté en la entrada $i$ al momento $t$, y de ser así, $z$ almacena el estado interno
de $M$. Ahora consideremos
\begin{equation}
  L_M = \{\langle s, i, t, z\rangle \;\;\;|\;\;\; con\;\;la\;\;entrada\;\;s\;\;tenemos\;\; z_{i,t} = z \;\;\;para\;\; M \}
\end{equation}
Simulando $M$ tendremos que $L_M \in EXP \subseteq P/poli$. Utilizando
circuitos de tamaño polinomial para $L_M$, podemos construir un circuito
 de tamaño polinomial $C$  de múltiple salida, tal que $C(\langle s,i,t\rangle)$ = $z$. Como buscábamos en $(1)$, decimos entonces que:\\
\begin{quote}
%  \center
$  s \in L \iff \exists C\;\; \forall i,t\;\; t.q.\;\; C(\langle s,i,t\rangle)\;\; acepta\;\;
si$\\
$C(\langle s,i-t,t-1\rangle),\;\;C(\langle s,i,t-1\rangle),\;\;C(\langle s,i+1,t-t1\rangle)\;\; y\;\; C(\langle s,1,2^{n^k}\rangle)\;\; aceptan.$
\end{quote}
Por lo tanto si $EXP \subseteq P/poli$, entonces $EXP = \Sigma^p_2$.
\end{document}
