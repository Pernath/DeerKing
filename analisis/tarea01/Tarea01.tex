%Especificacion
\documentclass[12pt]{article}

%Paquetes
\usepackage[left=2cm,right=2cm,top=3cm,bottom=3cm,letterpaper]{geometry}
\usepackage{lmodern}
\usepackage[T1]{fontenc}
\usepackage[utf8]{inputenc}
\usepackage[spanish,activeacute]{babel}
\usepackage{mathtools}
\usepackage{amssymb}
\usepackage{enumerate}
\usepackage{amsmath}
%\usepackage{graphicx}
%\graphicspath { {tarea01/media/} }
%\usepackage{pifont}
\newenvironment{boenumerate}
               {\begin{enumerate}\renewcommand\labelenumi{\textbf{\theenumi.}}}
               {\end{enumerate}}
               
\newcommand{\suchthat}{\mathrel{\mathop\supset}\kern-4.0pt$-$\kern-1.0pt$-~$}
               
               
%Preambulo
\title{Análisis de algoritmmos \\ Tarea 1}
\author{Carlos Gerardo Acosta Hernández \\ Andrea Itzel González Vargas}
\date{\today \\ Facultad de Ciencias UNAM}

\begin{document}
\maketitle

\section*{Ejercicios}

\begin{boenumerate}
\item 
  \begin{enumerate}
  \item $n^2$
    \begin{enumerate}
    \item Con $2n$ de entrada:
      \begin{align}
        (2n)^2 & = 2^2 \cdot n^2 \\ & = 4 \cdot n
      \end{align}
      Se vuelve cuatro veces más lento.
    \item Con $n+1$ de entrada:
      \begin{align}
        (n+1)^2 = n^2+2n+1
      \end{align}
      Aumenta en tiempo, un adicional $2n+1$.
    \end{enumerate}
  \item $n^3$
    \begin{enumerate}
    \item Con $2n$ de entrada:
      \begin{align}
        (2n)^3 & = 2^3 \cdot n^3 \\ & = 8 \cdot n
      \end{align}
      Se vuelve ocho veces más lento.
    \item Con $n+1$ de entrada:
      \begin{align}
        (n+1)^3 &= n\cdot(n^2+2n+1) \\
        &= n^3+2n^2+n
      \end{align}
      Aumenta en tiempo, un adicional $2n^2+n$.
    \end{enumerate}
  \item $100n^2$
    \begin{enumerate}
    \item Con $2n$ de entrada:
      \begin{align}
        100(2n)^2 &= 100(2^2n^2) \\
        &= 4 \cdot 100n^2 
      \end{align}
      El algoritmo se vuelve cuatro veces más lento.
    \item Con $n+1$ de entrada:
      \begin{align}
        100(n+1)^2 &= 100(n^2+2n+1) \\
        &= 100n^2+100(2n+1) \\
        &= 100n^2+200n+100
      \end{align}
      Por lo que se adiciona al tiempo original 200n+100.
    \end{enumerate}
  \item $n\;log\;n$
    \begin{enumerate}
    \item Con $2n$ de entrada:
      \begin{align}
        2n\cdot log(2n) &= 2n\cdot(log(2)+log(n)) \\
        &= 2n \cdot log_2(2) + 2n \cdot log(n) \\
        &= 2n \cdot 1 + 2n \cdot log(n) \\
        &= 2n \cdot log(n) + 2n 
      \end{align}
      Por lo que aumenta en tiempo al doble más un adicional $2n$.
    \item Con $n+1$ de entrada:
      \begin{align}
        (n+1) \cdot log(n+1) &= log\big((n+1)^{(n+1)}\big) \\
        &= log\big((n+1)^n \cdot (n+1)\big) \\
        &= log\big((n+1)^n\big) + log(n+1) \\
        &= log((n+1)^n) + log\Big((n+1) \cdot \frac{n^n}{n^n}\Big) \\
        &= n\;log(n+1) + log(n+1) + log\Big(\frac{n^n}{n^n}\Big) \\
        &= n\;log(n+1) + log(n+1) + log(n^n) - log(n^n) \\
        &= n\;log(n+1) + log(n+1) + n\;log(n) - n\;log(n) \\
        &= n\;logn + log(n+1) + n\cdot\big(log(n+1)-log(n)\big)
      \end{align}
      Por lo que aumenta en tiempo un adicional $\;\;\;log(n+1) + n\cdot\big(log(n+1)-log(n)\big)$
    \end{enumerate}
  \item $2^n$
    \begin{enumerate}
    \item Con $2n$ de entrada:
      \begin{align}
        2^{(2n)} = (2^n)^2
      \end{align}
      Por lo que el nuevo tiempo será la potencia cuadrada del original.
    \item Con $n+1$ de entrada:
      \begin{align}
        2^{(n+1)} = 2 \cdot 2^n
      \end{align}
      Aumenta al doble del original.
    \end{enumerate}
  \end{enumerate}
\item
  Primero debemos calcular el total de operaciones que puede realizar la computadora en una hora:
  \begin{equation}
    10^{10} \times (60 \cdot 60) = 3600 \times 10^{10} = 3.6 \times 10^{13} op/h
  \end{equation}
  Ahora simplemente igualamos el tiempo de ejecución de los algoritmos con el total de operaciones
  que se pueden realizar para hallar la entrada máxima de cada uno, computable en una hora.
  \begin{enumerate}
  \item $n^2 = 3.6 \times 10^{13} \Rightarrow n = \sqrt{3600 \times 10^{10}} = 60 \times 10^{5} = 6000000$ elementos.
  \item $n^3 = 3.6 \times 10^{13} \Rightarrow n = \sqrt[3]{3600 \times 10^{10}} \approx 33019$ elementos\footnote{No contamos las fracciones de elementos: $n = 33019.27248894625$}.
  \item $100n^2 = 3.6 \times 10^{13} \Rightarrow n = \frac{\sqrt{3600 \times 10^{10}}}{100} = 60 \times 10^{3} = 60000$ elementos.
  \item $n\;log\;n = 3.6 \times 10^{13} \Rightarrow n = $
  \item $2^n = 3.6 \times 10^{13} \Rightarrow n = log_2(3.6 \times 10^{13}) \approx 45$ elementos\footnote{No consideramos elementos que no sean enteros: n = 45.033062140090664}.
  \item $2^{2^n} = 3.6 \times 10^{13} \Rightarrow n = log_2(log_2(3.6 \times 10^{13})) \approx log_2(45) \approx 5$ elementos\footnote{Una vez más consideramos sólo la parte entera: n = 5.49291267570108}.
  \end{enumerate}
\item Comencemos recordando que para las funciones exponenciales respecto de las polinomiales, se cumple:
  $\forall r,d \suchthat r > 1 \Rightarrow n^d \in O(r^n)$,
  por lo que podemos determinar que $f_1(n) \in O(f_4(n))$.

  De manera directa, podemos decir que $f_4(n) \in O(f_5(n))$, para los positivos a partir del 1. 
  
  Ahora veamos que $f_1(n) = n^{2.5} = n^{2+0.5} = n^2 \cdot n^{\frac{1}{2}}$, comparándolo con $f_6(n)$, al tener que $log_2(n) \leq n^{\frac{1}{2}} \Rightarrow n^2logn \leq n^{2.5}$, y notando que para valores mayores a $4$, $f_1(n)$ es mayor siempre que $f_6(n)$, podemos concluir que $f_6(n) \in O(f_1(n))$.
  
  Hemos visto en clase que $n^2 \in O(n^2logn)$ y que $n \in O(n^2)$. Por transitividad, podemos ver que 
  $f_3(n) \in O(f_6(n))$. De cualquier manera, podemos ver que a partir de valores como 3, $f_6$ será siempre mayor a $f_3$.
  

  Finalmente, es claro que al no estar definidas las raíces negativas en los reales, desde el 0 a valores positivos, la función $f_3$ es mayor a $f_2$.

  Así pues, queda en orden creciente nuestra lista de funciones como sigue:
  $f_2,f_3,f_6,f_1,f_4,f_5$.
\item
  \begin{itemize}
    \item Primero comparamos $g_6(n) = 2^{2^n}$ y $g_7(n) = 2^{n^2}$: \\
      Sacamos el logaritmo de ambas funciones, $log(g_6(n)) = 2^n$ y $log(g_7(n)) = n^2$, y como sabemos que $n^2 = O(2^n)$, entonces $g_7(n) = O(g_6(n))$.
    \item Ahora comparamos $g_2(n) = 2^n$ y $g_7(n) = 2^{n^2}$: \\
      Como en el caso anterior, sacamos el logaritmo de ambas funciones, $log(g_2(n)) = n$ y $log(g_7(n)) = n^2$, y como sabemos que $n = O(n^2)$, entonces $g_2(n) = O(g_7(n))$.
    \item Comparamos $g_2(n) = 2^n$ y $g_5(n) = n^{log\ n}$: \\
      Sacamos el logaritmo de ambas funciones, $log(g_2(n)) = n$ y $log(g_5(n)) = log(n^{log\ n}) = log\ n * log\ n = (log\ n)^2$, y como sabemos que $(log\ n)^2 = O(n)$, entonces $g_5(n) = O(g_2(n))$.
    \item Comparamos $g_3(n) = n(log\ n)^3$ y $g_5(n) = n^{log\ n}$: \\
      Dividimos ambas funciones entre $n$, cuyo resultado es $g_3(n)/n = (log\ n)^3$ y $g_5(n) = n^{log\ n - 1}$. Tenemos que $g_3(n)/n$ es de orden logarítmico y $g_5(n)/n$ es de orden polinomial, entonces $g_3(n)/n = O(g_5(n))/n$ y por lo tanto $g_3(n) = O(g_5(n))$.
    \item Comparamos $g_3(n) = n(log\ n)^3$ y $g_4(n) = n^{4/3}$: \\
      Dividimos entre $n$ y sacamos la raíz cúbica de ambas funciones, $f_1 = \sqrt[3]{\frac{g_3(n)}{n}} = log\ n$ y $f_2 = \sqrt[3]{\frac{g_4(n)}{n}} = n^{1/9}$. \\
      Sacamos el logaritmo de ambas funciones y nos queda $log\ f_1 = log(log\ n))$ y $log\ f_2 = log(n^{1/9}) = \frac{1}{9}log\ n$. \\
      Sustituimos $z = log\ n$, con lo cual tenemos $log\ f_1 = log\ z$ y $log\ f_2 = \frac{1}{9}z$. Tenemos que para $z > 1$, $log\ z > \frac{1}{9}z$, y por lo tanto $g_4(n) = O(g_3(n))$.
    \item Comparamos $g_1(n) = 2^{\sqrt{log\ n}}$ con $g_4(n) = n^{4/3}$: \\
      Tenemos que $log(g_1(n)) = log(2^{\sqrt{log\ n}}) = \sqrt{log\ n}$ y $log(g_4(n)) = log(n^{4/3}) = \frac{4}{3}log\ n = log\ n + \frac{1}{3} log\ n$. Sabemos que $log\ n > \sqrt{log\ n}$, entonces $log\ n + \frac{1}{3} log\ n > \sqrt{log\ n}$ y por lo tanto $g_1 = O(g_4)$. \\
        Concluimos que nuestro orden creciente de funciones queda de la siguiente manera: \\
        $g_6,g_7,g_2,g_5,g_3,g_4,g_1$.
        
  \end{itemize}  
\item
  \begin{enumerate}
  \item Falso. Abusando un poco de la definición de la cota justa, supongamos que para toda $n$, tenemos que $f(n) = 2$ y 
    $g(n) = 1$. Considerando una $c \geq 2$, es fácil ver que $f(n) \leq c \cdot g(n) \Rightarrow f(n) \in O(g(n))$, por lo que al aplicar el logaritmo, obtenemos $log_2\;f(n) = log_2\; 2 = 1$ y $log_2\;g(n) = log_2\; 1 = 0$. Con ese resultado, es imposible hallar una constante $c$ que siga cumpliendo la definición para la cota superior asintótica ($f(n) > c\cdot g(n) \;\forall n$, en este ejemplo).
  \item Falso. Suponiendo que $f(n) = 2n$ y $g(n) = n$. Considerando una $c \geq 2$, es fácil ver que $f(n) \leq c \cdot g(n) \Rightarrow f(n) \in O(g(n))$. Sin embargo, $2^{f(n)} \in O(2^{g(n)})$ no se cumplirá
    pues $2^{2n} \geq c \cdot 2^n$; de hecho, con una $n$ lo suficientemente grande, la igualdad no ocurrirá. Es decir que para cualquier constante que elijamos, la relación solicitada no se mantendrá para toda $n$.
  \item Verdadero. Tenemos que $f(n) \leq c \cdot g(n)$ p.a. $c > 0$, por el crecimiento
    de una función cuadrática en el cuadrante real positivo, el orden se preservará para
    $(f(n)^2) \leq c \cdot (g(n))^2$.
  \end{enumerate}
\item
  \begin{enumerate}
  \item Primero calculemos el número de operaciones que son necesarias en una ejecución del algoritmo. Notemos que son necesarias $j-i$ para ejecutar la primera línea después del segundo \texttt{for};
    ocurrirá $n^2$ veces. Podemos expresar el total de sumas que operan de la manera siguiente:
    \begin{align}
      \sum_{i=1}^{n}(\sum_{i=1}^{j} j-i)
    \end{align}
    Sin embargo, hemos de considerar la operación de alojamiento de la suma resultante, así que sumamos finalmente una $n^2$. Obtenemos entonces:
    \begin{align}
      (\sum_{i=1}^{n}(\sum_{i=1}^{j} j-i)) + n^2 &= \frac{1}{6}(n-1)(n+1) + n^2\\
      &= \frac{n^3}{6} + n^2 - \frac{n}{6}
    \end{align}
    Ahora es más claro escoger la función $f$ que se nos solicita para la cota superior como $f(n) = n^3$. Veamos que  $\frac{n^3}{6} + n^2 - \frac{n}{6} \in O(n^3)$. Si escogemos como $n_0 = 1$, y nuestra constante $c = 1$, tendremos que para toda $n \geq n_0$, $\frac{n^3}{6} + n^2 - \frac{n}{6} \leq n^3\cdot 1$. \\
    
    \textbf{Inducción sobre $n$:}\\
    Caso base: $n=n_0$, $\frac{(1)^3}{6} + (1)^2 + \frac{(1)}{6} = \frac{1}{6}+1-\frac{1}{6} = 1 = (1)^3$. \\
    Hipótesis de inducción: Supongamos que se cumple para $n = m$; m > $n_0$.\\
    Ahora $n =  m+1$
    P.d. $\frac{(m+1)^3}{6} + (m)^2 + \frac{(m+1)}{6} < (m+1)^3$
    \begin{align}
      \frac{(m+1)^3}{6} + (m+1)^2 + \frac{(m+1)}{6} &= \frac{m^3 + 3m^2+3m+1}{6} + m^2 + 2m +1 - \frac{m+1}{6} \\ 
      &= \frac{m^3 + 3m^2+2m}{6} + m^2 + 2m +1 \\
      &= \frac{m^3 + 9m^2+14m+6}{6}\\
      &= \frac{m^3}{6} + \frac{9m^2}{6} + \frac{14m}{6} + 1
    \end{align}
    Comparandolo con $(m+1)^3 = m^3 +  3m^2 + 3m+ 1$. Es fácil ver que:
    \begin{itemize}
    \item $\frac{m^3}{6} < m^3$
    \item $\frac{9m^2}{6} < 3m^2$
    \item $\frac{14m}{6} < 3m$
    \item $1 = 1$           
    \end{itemize}
    $\Rightarrow \frac{(m+1)^3}{6} + (m)^2 + \frac{(m+1)}{6} < (m+1)^3$ \;\;\;\;Q.E.D. \\
      \begin{center}
        $\therefore \frac{n^3}{6} + n^2 - \frac{n}{6} \in O(n^3)$
        \end{center}
  \item
  \item
  \end{enumerate}
\end{boenumerate}

\end{document}
