%Especificacion
\documentclass[12pt]{article}

%Paquetes
\usepackage[left=2cm,right=2cm,top=3cm,bottom=3cm,letterpaper]{geometry}
\usepackage{lmodern}
\usepackage[T1]{fontenc}
\usepackage[utf8]{inputenc}
\usepackage[spanish,activeacute]{babel}
\usepackage{mathtools}
\usepackage{amssymb}
\usepackage{enumerate}
\usepackage{amsmath}
%\usepackage{graphicx}
%\graphicspath { {tarea01/media/} }
%\usepackage{pifont}
\newenvironment{boenumerate}
               {\begin{enumerate}\renewcommand\labelenumi{\textbf{\theenumi.}}}
               {\end{enumerate}}

%Preambulo
\title{Análisis de algoritmmos \\ Tarea 1}
\author{Carlos Gerardo Acosta Hernández \\ Andrea Itzel González Vargas}
\date{\today \\ Facultad de Ciencias UNAM}

\begin{document}
\maketitle

\section*{Ejercicios}

\begin{boenumerate}
\item 
  \begin{enumerate}
  \item $n^2$
    \begin{enumerate}
    \item Con $2n$ de entrada:
      \begin{align}
        (2n)^2 & = 2^2 \cdot n^2 \\ & = 4 \cdot n
      \end{align}
      Se vuelve cuatro veces más lento.
    \item Con $n+1$ de entrada:
      \begin{align}
        (n+1)^2 = n^2+2n+1
      \end{align}
      Aumenta en tiempo, un adicional $2n+1$.
    \end{enumerate}
  \item $n^3$
    \begin{enumerate}
    \item Con $2n$ de entrada:
      \begin{align}
        (2n)^3 & = 2^3 \cdot n^3 \\ & = 8 \cdot n
      \end{align}
      Se vuelve ocho veces más lento.
    \item Con $n+1$ de entrada:
      \begin{align}
        (n+1)^3 &= n\cdot(n^2+2n+1) \\
        &= n^3+2n^2+n
      \end{align}
      Aumenta en tiempo, un adicional $2n^2+n$.
    \end{enumerate}
  \item $100n^2$
    \begin{enumerate}
    \item Con $2n$ de entrada:
      \begin{align}
        100(2n)^2 &= 100(2^2n^2) \\
        &= 4 \cdot 100n^2 
      \end{align}
      El algoritmo se vuelve cuatro veces más lento.
    \item Con $n+1$ de entrada:
      \begin{align}
        100(n+1)^2 &= 100(n^2+2n+1) \\
        &= 100n^2+100(2n+1) \\
        &= 100n^2+200n+100
      \end{align}
      Por lo que se adiciona al tiempo original 200n+100.
    \end{enumerate}
  \item $n\;log\;n$
    \begin{enumerate}
    \item Con $2n$ de entrada:
      \begin{align}
        2n\cdot log(2n) &= 2n\cdot(log(2)+log(n)) \\
        &= 2n \cdot log_2(2) + 2n \cdot log(n) \\
        &= 2n \cdot 1 + 2n \cdot log(n) \\
        &= 2n \cdot log(n) + 2n 
      \end{align}
      Por lo que aumenta en tiempo al doble más un adicional $2n$.
    \item Con $n+1$ de entrada:
      \begin{align}
        (n+1) \cdot log(n+1) &= log\big((n+1)^{(n+1)}\big) \\
        &= log\big((n+1)^n \cdot (n+1)\big) \\
        &= log\big((n+1)^n\big) + log(n+1) \\
        &= log((n+1)^n) + log\Big((n+1) \cdot \frac{n^n}{n^n}\Big) \\
        &= n\;log(n+1) + log(n+1) + log\Big(\frac{n^n}{n^n}\Big) \\
        &= n\;log(n+1) + log(n+1) + log(n^n) - log(n^n) \\
        &= n\;log(n+1) + log(n+1) + n\;log(n) - n\;log(n) \\
        &= n\;logn + log(n+1) + n\cdot\big(log(n+1)-log(n)\big)
      \end{align}
      Por lo que aumenta en tiempo un adicional $\;\;\;log(n+1) + n\cdot\big(log(n+1)-log(n)\big)$
    \end{enumerate}
  \item $2^n$
    \begin{enumerate}
    \item Con $2n$ de entrada:
      \begin{align}
        2^{(2n)} = (2^n)^2
      \end{align}
      Por lo que el nuevo tiempo será la potencia cuadrada del original.
    \item Con $n+1$ de entrada:
      \begin{align}
        2^{(n+1)} = 2 \cdot 2^n
      \end{align}
      Aumenta al doble del original.
    \end{enumerate}
  \end{enumerate}
\item
  Primero debemos calcular el total de operaciones que puede realizar la computadora en una hora:
  \begin{equation}
    10^{10} \times (60 \cdot 60) = 3600 \times 10^{10} = 3.6 \times 10^{13} op/h
  \end{equation}
  Ahora simplemente igualamos el tiempo de ejecución de los algoritmos con el total de operaciones
  que se pueden realizar para hallar la entrada máxima de cada uno, computable en una hora.
  \begin{enumerate}
  \item $n^2 = 3.6 \times 10^{13} \Rightarrow n = \sqrt{3600 \times 10^{10}} = 60 \times 10^{5} = 6000000$ elementos.
  \item $n^3 = 3.6 \times 10^{13} \Rightarrow n = \sqrt[3]{3600 \times 10^{10}} \approx 33019$ elementos\footnote{No contamos las fracciones de elementos: $n = 33019.27248894625$}.
  \item $100n^2 = 3.6 \times 10^{13} \Rightarrow n = \frac{\sqrt{3600 \times 10^{10}}}{100} = 60 \times 10^{3} = 60000$ elementos.
  \item $n\;log\;n = 3.6 \times 10^{13} \Rightarrow n = $
  \item $2^n = 3.6 \times 10^{13} \Rightarrow n = log_2(3.6 \times 10^{13}) \approx 45$ elementos\footnote{No consideramos elementos que no sean enteros: n = 45.033062140090664}.
  \item $2^{2^n} = 3.6 \times 10^{13} \Rightarrow n = log_2(log_2(3.6 \times 10^{13})) \approx log_2(45) \approx 5$ elementos\footnote{Una vez más consideramos sólo la parte entera: n = 5.49291267570108}.
  \end{enumerate}
\item
\item
\item
\item
\end{boenumerate}

\end{document}
