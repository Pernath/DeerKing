% Especificacion
\documentclass[12pt]{article}

% Paquetes
\usepackage[left=2cm,right=2cm,top=3cm,bottom=3cm,letterpaper]{geometry}
\usepackage{lmodern}
\usepackage[T1]{fontenc}
\usepackage[utf8]{inputenc}
\usepackage[spanish,activeacute]{babel}
\usepackage{mathtools}
\usepackage{amssymb}
\usepackage{enumerate}
\usepackage{enumitem}
% \usepackage{graphicx}
% \graphicspath { {tarea01/media/} }
% \usepackage{pifont}

% Preambulo
\title{Análisis de algoritmmos \\ Tarea 1}
\author{Karla Adriana Esquivel Guzmán \\ Luis Pablo Mayo Vega}
\date{\today \\ Facultad de Ciencias UNAM}

\begin{document}
\maketitle

\section*{Ejercicios}
\subsubsection*{1.}
\subsubsection*{2.}
\begin{enumerate}[label=(\alph*)]
\item $n^2$\\
  Sabemos que la computadora con la que contamos realiza $10^{10}$ operaciones por segundo y que el algoritmo toma tiempo $n^2$. Entonces, para una entrada de tamaño $\sqrt{10^{10}}$, el algoritmo se ejecutaría en exactamente 1 segundo.\\
  Como una hora tiene 3600 segundos, buscamos una entrada $n$ que tome $10^{10}\times 3600$ operaciones.
  \begin{multline*}
    \\x = 10^{10} \times 3600 = 36 \times 10^{12}\\
    a = \sqrt{10^{10}} = 100,000\\
    b = \sqrt{3600}= 60\\
    n = a \times b = 6,000,000\\
    n^2 = 36 \times 10^{12} = x\\   
  \end{multline*}
  $\therefore n = 6,000,000$   
\item $n^3$\\
  Siguiendo el mismo método que en \textit{(a)}, la entrada de tamaño $\sqrt[3]{10^{10}}$ le tomaría 1 segundo para ejecutarse y el factor de tiempo también estaría determinado por $\sqrt[3]{3600}$
  \begin{multline*}
    \\x = 36 \times 10^{12}\\
    a = \sqrt[3]{10^{10}} \approx 2154v\\
    b = \sqrt{3600} \approx 15 \\
    n = a \times b \approx 32310\\
    n^3 \approx 34 \times 10^{12}\\   
  \end{multline*}
  Como aún hay una diferencia notable entre la $n^3$ con esta entrada y el número total de operaciones posibles en una hora, podemos aproximarla más con un algoritmo como el siguiente.
\begin{verbatim}
def aproxima_n(i, x):
    t = 3600^(1/3) 
    b = (i*t)^3
    while x > b:
        b  = (i*t)^3
        if b > x:
            return (i-1)*t
        i+=1
\end{verbatim}
  donde $i = \sqrt[3]{10^{10}} \;,\; x=36 \times 10^{12}$ y  obtenemos como resultado $n = 33019$
\item $100n^2$ \\
  De manera análoga a (a) pero le dividimos la constante 100 
  \begin{multline*}
    \\x = 36 \times 10^{12}\\
    a = \frac{\sqrt{10^{10}}}{10} = 10,000\\
    b = \sqrt{3600}= 60\\
    n = a \times b = 600,000\\
    100n^2 = 36 \times 10^{12} = x\\   
  \end{multline*}

\item $n log n$
  \begin{multline*}
    \\ x = 36 \times 10^{12}\\
    n log n = x \implies log n = \frac{x}{n}\\
    \implies 2^{\frac{x}{n}} = n \\
  \end{multline*}
  n es casi tan grande como x ...%sorry
\item $2^n$\\
  Aplicando leyes de exponentes y logaritmos:
  \begin{multline*}
    \\ x  =  36 \times 10^{12}  =  2^n \\
    \implies \log({10^{10} \times 3600} ) =  n\\
    n =  45\\
    \therefore n = 45\\
  \end{multline*}
\item $2^{2^{n}}$ \\
  Considerando que $log(36 \times 10^{12}) = 45$ debemos encontrar una \textit{n} tal que $2^n \leq 45$\\
  \begin{multline*}
    \\ x = 36 \times 10^{12}\\
    \log(45) \approx 5
    \implies 2^{2^{5}} \approx x \\
    \therefore n = 5 \\
  \end{multline*}
\end{enumerate}
\subsubsection*{3.}
\subsubsection*{4.}
\subsubsection*{5.}
\subsubsection*{6.}

\end{document}
